\myparagraph{Purpose}
The main feature of \textit{AutomatedSOS} is to check the health status of the user and it has to detect any critical situation.
\textit{AutomatedSOS} monitors the health status of the subscribed customers and, when such parameters are below certain thresholds, sends to the location of the customer to an ambulance, guaranteeing a reaction time of less than 5 seconds from the time the parameters are below the threshold.

\myparagraph{Scenario 1}
For a couple of days Vittorio felt very tired and affected of sickness. While he was walking in his house his heartbeat goes down and he lie down on the floor. Luckily he had installed \textit{AutomatedSOS} application.
The application detected a critical situation, it managed to track Vittorio position and it called the ambulace.
The ambulace arrived very quickly and luckily the paramedics with a cardiac massage saved Vittorio that was carried to the hospital.

\myparagraph{Scenario 2}
Cristiano is a professional runner. He decides to install on his phone \textit{AutomatedSOS} to keep trace of his health status and avoid any possible critical situation when he runs.
One day, while he was running, \textit{AutomatedSOS} went in an alerted status but after only 1 second the application came back in a normal status and stayed in the normal status all run duration.
Probably it was an abnormal device measure of life value, so \textit{AutomatedSOS} did not call the ambulance.

\myparagraph{Use Case}
The \textit{Critical Situation} use case is analyzed in Table \ref{table:criticalSituation}.

\myparagraph{Functional requirements}
\begin{enumerate}
  \item The system must guarantee a reaction time of less than 5 seconds from the time it is in an alerted status;
  \item The system must send the location of the customer to an ambulance;
  \item The system must be in an alerted status when maximum pressure value of the customer is more than 170 mmHg and minimum pressure value is more than 100 mmHg;
  \item The system must be in an alerted status when the heartbeat is lower than 45 bmp or it is higher than 120 bpm;
  \item If the system goes in an alerted status it has to increase the life parameters detection frequency.
\end{enumerate}

\begin{center}
\begin{table}
\begin{tabular}{ | l | p{0.75\linewidth} | }
  \hline
    Actor & \textbf{System}, \textbf{User} \\ \hline
    Goal & \textbf{[G.8]} \\ \hline
    Input Condition & The system goes in an alerted status \\ \hline
    Event Flow & \begin{minipage}[t]{0.7\textwidth}
      \begin{enumerate}
        \item The \textbf{System} gets the GPS position of the \textbf{User};
        \item The \textbf{System} increases parameters detection with a frequence of 3 detection per second;
        \item The \textbf{System} shows an alert message on the \textbf{User} device (smartwatch or similar).
      \end{enumerate}
    \smallskip
  \end{minipage} \\ \hline
  Output Condition & The \textbf{System} calls an ambulance and it sends the location to the called ambulance if it is in an alert status from 3 seconds. \\ \hline
  Exceptions & \begin{minipage}[t]{0.7\textwidth}
    \begin{itemize}
      \smallskip
      \item If functional requirement 2 is not satisfied the \textbf{System} notifies the ambulance with a detected position error and it sends the last deteted position;
      \item If functional requirements 3 and 4 are not satisfied the \textbf{System} notifies the \textbf{User} with a warning message and it invites the \textbf{User} to check the status and the conncetion of his/her devices.
    \end{itemize}
    \smallskip
  \end{minipage}  \\ \hline
\end{tabular}
\caption{\textit{Critical Situation} use case}
\label{table:criticalSituation}
\end{table}
\end{center}
