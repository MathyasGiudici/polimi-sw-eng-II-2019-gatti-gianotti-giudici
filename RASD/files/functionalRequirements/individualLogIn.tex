\myparagraph{Purpose}
The main goal of the login feature is to allow the access to one of the services of \textit{Data4Help} to any registered user. To access to an application the user has to fill out the credential form where e-mail and password are required. Moreover, there is a \textit{Forgot password?} section where a user could recover his/her password via e-mail. An e-mail is sended to the user with a temporary password that user will change once logged in.\\
Moreover, at the first login in one of the application the \textit{Individual User} must associate to the system one device (like smartwatch or similar) to allow the system to trace his/her data. This process is very important in \textit{AutomatedSOS} application.

\myparagraph{Scenario 1}
Francesca loves running. When she has heard about \textit{Track4Run} application she downloaded it immediately. Her friend Clara told her about a charity run for the following weekend, so Francesca opened \textit{Track4Run}, she clicked on \textit{Log in}. She inserted her e-mail address and password and clicked on the login button. Everything was correct, so she entered in the system and could enroll the run.

\myparagraph{Scenario 2}
One year ago Tommaso, Aldo's grandchild, installed on Aldo’s phone \textit{AutomatedSOS}. Yesterday, Aldo bought a new phone, he downloaded the app but he forgot his password so he couldn’t log in the application. He clicked on \textit{Forgot password?} he inserted his e-mail address and clicked on \textit{Restore my password}. He received a mail with the provisional password and he became able to access to the system. After the access he was forced to change his provisional password into a new one.

\myparagraph{Scenario 3}
According to the scenario presented in Section \ref{aSOSsignin} where we told about the registration of Sara's grandmother.\\
Sara helps her grandmother to do the first log in \textit{AutomatedSOS}. After inserting of credentials and a succesfull login, Sara has to match the grandma's smartwatch with the application. \textit{AutomatedSOS} has a wizard to help users: Sara turns on the bluetooth in each devices (smartwatch and phone); after the phone matching with the smartwatch via bluetooth (system matching), Sara has to do the matching with \textit{AutomatedSOS}. On the smartwatch's screen there is a six digits number and Sara puts this number in \textit{AutomatedSOS} application and she clicks on \textit{Done} button. After that smartwatch and phone are matched and \textit{AutomatedSOS} is able to watch the health status of Sara's grandmother.

\myparagraph{Use Case}
The \textit{Generic Individual Log In} use case is analyzed in Table \ref{table:genericIndividualLogInInTable}.\\
The \textit{First Individual Log In} use case is analyzed in Table \ref{table:firstIndividualLogInInTable}.

\myparagraph{Functional requirements}
\begin{enumerate}
  \item The \textbf{Individual User} must be already registered in the system in order to log in successfully;
  \item The \textbf{Individual User} has to remember his/her e-mail address and password in order to log in successfully;
  \item The password inserted by the \textbf{Individual User} must correspond with the e-mail address;
  \item If the \textbf{Individual User} inserts wrong credential could not be able to access to the system;
  \item If the \textbf{Individual User} clicks on \textit{Forgot password?}, the system sends a new password to the \textbf{Individual User} e-mail address if and only if the e-mail address is valid and registered to the system;
  \item After a password restoring through \textit{Forgot password?} operations, the system must allow the access with the temporary password and then it has to force the \textbf{Individual User} to change the temporary password into a new one;
  \item (First access only) The inserted matching number by the \textbf{Individual User} must correspond with the visualized on the device's screen.
\end{enumerate}

\begin{center}
\begin{table}
\begin{tabular}{ | l | p{0.75\linewidth} | }
  \hline
    Actor & \textbf{Individual user} \\ \hline
    Goal & \textbf{[G.2]} \\ \hline
    Input Condition & The \textbf{Individual User} is already registered to the system and want to log in \\ \hline
    Event Flow & \begin{minipage}[t]{0.7\textwidth}
      \begin{enumerate}
        \item The \textbf{Individual User} open one of the applications (\textit{AutomatedSOS} or \textit{Track4Run});
        \item The \textbf{Individual User} clicks on \textit{Log In} button;
        \item The \textbf{Individual User} fills in the fields with his/her e-mail address and password;
        \item The \textbf{Individual User} clicks on the login button.
      \end{enumerate}
    \smallskip
  \end{minipage} \\ \hline
  Output Condition & The system allows the login of the \textbf{Individual User} and loads the dashboard of the \textbf{Individual User}. \\ \hline
  Exceptions & \begin{minipage}[t]{0.7\textwidth}
    \begin{itemize}
      \smallskip
      \item If functional requirements 1 or 3 are not satisfied the system notifies the \textbf{Individual User} with an error message and the process goes back to step 3;
      \item If the \textbf{Individual User} inserts wrong credentials for three times the system notifies him/her with an e-mail.
    \end{itemize}
    \smallskip
  \end{minipage}  \\ \hline
\end{tabular}
\caption{\textit{Generic Individual Log In} use case}
\label{table:genericIndividualLogInInTable}
\end{table}
\end{center}

\begin{center}
\begin{table}
\begin{tabular}{ | l | p{0.75\linewidth} | }
  \hline
    Actor & \textbf{Individual user} \\ \hline
    Goal & \textbf{[G.2]} \\ \hline
    Input Condition & The \textbf{Individual User} is already registered to the system and want to log in \\ \hline
    Event Flow & \begin{minipage}[t]{0.7\textwidth}
      The first part of the event flow is already explained in Table \ref{table:genericIndividualLogInInTable}.
      \smallskip
      \begin{enumerate}
        \item The \textbf{Individual User} has to match a device to \textit{AutomatedSOS} or \textit{Track4Run} application;
        \item The \textbf{Individual User} switchs on the bluetooth in each devices;
        \item The \textbf{Individual User} connects the mobile phone with the health device (system connection via bluetooth);
        \item The \textbf{Individual User} inserts the six digits number visualized on the screen of the health device;
        \item The \textbf{Individual User} clicks on the \textit{Done} button;
      \end{enumerate}
    \smallskip
  \end{minipage} \\ \hline
  Output Condition & The system allows the login of the \textbf{Individual User} and loads the dashboard of the \textbf{Individual User}. \\ \hline
  Exceptions & \begin{minipage}[t]{0.7\textwidth}
    All already explained exceptions in Table \ref{table:genericIndividualLogInInTable} are still valid.
    \begin{itemize}
      \smallskip
      \item If functional requirements 7 is not satisfied the system notifies the \textbf{Individual User} with an error message and the process goes back to step 2;
    \end{itemize}
    \smallskip
  \end{minipage}  \\ \hline
\end{tabular}
\caption{\textit{First Individual Log In} use case}
\label{table:firstIndividualLogInInTable}
\end{table}
\end{center}
