\myparagraph{Purpose}
The main goal of the login feature is to allow the access to one of the services of \textit{Data4Help} to any registered user. To access to an application the user has to fill out the credential form where e-mail and password are required. Moreover, there is a \textit{Forgot password?} section where a user could recover his/her password via e-mail. An e-mail is sended to the user with a temporary password that user will change once logged in.

\myparagraph{Scenario 1}
Francesca loves running. When she has heard about \textit{Track4Run} application she downloaded it immediately. Her friend Clara told her about a charity run for the following weekend, so Francesca opened \textit{Track4Run}, she clicked on \textit{Log in}. She inserted her e-mail address and password and clicked on the login button. Everything was correct, so she entered in the system and could enroll the run.

\myparagraph{Scenario 2}
One year ago Tommaso, Aldo's grandchild, installed on Aldo’s phone \textit{AutomatedSOS}. Yesterday, Aldo bought a new phone, he downloaded the app but he forgot his password so he couldn’t log in the application. He clicked on \textit{Forgot password?} he inserted his e-mail address and clicked on \textit{Restore my password}. He received a mail with the provisional password and he became able to access to the system. After the access he was forced to change his provisional password into a new one.

\myparagraph{Use Case}
The Individual Log In is analyzed in Table \ref{table:individualLogInInTable}.

\myparagraph{Functional requirements}
\begin{enumerate}
  \item The \textbf{Individual User} must be already registered in the system in order to log in successfully;
  \item The \textbf{Individual User} has to remember his/her e-mail address and password in order to log in successfully;
  \item The password inserted by the \textbf{Individual User} must correspond with the e-mail address;
  \item If the \textbf{Individual User} inserts wrong credential could not be able to access to the system;
  \item If the \textbf{Individual User} clicks on \textit{Forgot password?}, the system sends a new password to the \textbf{Individual User} e-mail address if and only if the e-mail address is valid and registered to the system;
  \item After a password restoring through \textit{Forgot password?} operations, the system must allow the access with the temporary password and then it has to force the \textbf{Individual User} to change the temporary password into a new one.
\end{enumerate}

\begin{center}
\begin{table}
\begin{tabular}{ | l | p{0.75\linewidth} | }
  \hline
    Actor & \textbf{Individual user} \\ \hline
    Goal & \textbf{[G.2]} \\ \hline
    Input Condition & The \textbf{Individual User} is already registered to the system and want to log in \\ \hline
    Event Flow & \begin{minipage}[t]{0.7\textwidth}
      \begin{enumerate}
        \item The \textbf{Individual User} open one of the applications (\textit{AutomatedSOS} or \textit{Track4Run});
        \item The \textbf{Individual User} clicks on \textit{Log In} button;
        \item The \textbf{Individual User} fills in the fields with his/her e-mail address and password;
        \item The \textbf{Individual User} clicks on the login button.
      \end{enumerate}
    \smallskip
  \end{minipage} \\ \hline
  Output Condition & The system allows the login of the \textbf{Individual User} and loads the dashboard of the \textbf{Individual User}. \\ \hline
  Exceptions & \begin{minipage}[t]{0.7\textwidth}
    \begin{itemize}
      \smallskip
      \item If the inserted e-mail address is never been registered to the system or the password doesn’t correspond with the email address, the system notifies the \textbf{Individual User} with an error message;
      \item If the \textbf{Individual User} inserts wrong credentials for three times the system notifies him/her with an e-mail.
    \end{itemize}
    \smallskip
  \end{minipage}  \\ \hline
\end{tabular}
\caption{Individual Log In use case}
\label{table:individualLogInInTable}
\end{table}
\end{center}
